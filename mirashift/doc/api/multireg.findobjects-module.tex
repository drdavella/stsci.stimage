%
% API Documentation for ImageShift
% Module multireg.findobjects
%
% Generated by epydoc 2.1
% [Thu Dec 15 12:26:52 2005]
%

%%%%%%%%%%%%%%%%%%%%%%%%%%%%%%%%%%%%%%%%%%%%%%%%%%%%%%%%%%%%%%%%%%%%%%%%%%%
%%                          Module Description                           %%
%%%%%%%%%%%%%%%%%%%%%%%%%%%%%%%%%%%%%%%%%%%%%%%%%%%%%%%%%%%%%%%%%%%%%%%%%%%

    \index{multireg.findobjects \textit{(module)}|(}
\section{Module multireg.findobjects}

    \label{multireg:findobjects}

%%%%%%%%%%%%%%%%%%%%%%%%%%%%%%%%%%%%%%%%%%%%%%%%%%%%%%%%%%%%%%%%%%%%%%%%%%%
%%                               Functions                               %%
%%%%%%%%%%%%%%%%%%%%%%%%%%%%%%%%%%%%%%%%%%%%%%%%%%%%%%%%%%%%%%%%%%%%%%%%%%%

  \subsection{Functions}

    \label{multireg:findobjects:build_LOGKernel}
    \index{multireg.findobjects \textit{(module)}!build\_LOGKernel \textit{(function)}}
    \vspace{0.5ex}

    \begin{boxedminipage}{\textwidth}

    \raggedright \textbf{build\_LOGKernel}(\textit{size}, \textit{sigma})

    \vspace{-1.5ex}

    \rule{\textwidth}{0.5\fboxrule}
    Build an LoG kernel of arbitrary size

    \vspace{1ex}

    \end{boxedminipage}

    \label{multireg:findobjects:center1d}
    \index{multireg.findobjects \textit{(module)}!center1d \textit{(function)}}
    \vspace{0.5ex}

    \begin{boxedminipage}{\textwidth}

    \raggedright \textbf{center1d}(\textit{region})

    \vspace{-1.5ex}

    \rule{\textwidth}{0.5\fboxrule}
    Compute the center of gravity of a 1-d array. Based on 
    'mpc\_getcenter' from IRAF imutil task 'center' in the cl.proto 
    package.

    \vspace{1ex}

    \end{boxedminipage}

    \label{multireg:findobjects:DEGTORAD}
    \index{multireg.findobjects \textit{(module)}!DEGTORAD \textit{(function)}}
    \vspace{0.5ex}

    \begin{boxedminipage}{\textwidth}

    \raggedright \textbf{DEGTORAD}(\textit{deg})

    \end{boxedminipage}

    \label{multireg:findobjects:discriminate_source}
    \index{multireg.findobjects \textit{(module)}!discriminate\_source \textit{(function)}}
    \vspace{0.5ex}

    \begin{boxedminipage}{\textwidth}

    \raggedright \textbf{discriminate\_source}(\textit{array}, \textit{labels}, \textit{idnum})

    \vspace{-1.5ex}

    \rule{\textwidth}{0.5\fboxrule}
    For source number 'idnum' from labels, determine whether it is a 
    positive feature or the edge of a larger source. It will return 0 for 
    an edge and 1 for a source.

    \vspace{1ex}

    \end{boxedminipage}

    \label{multireg:findobjects:find_center}
    \index{multireg.findobjects \textit{(module)}!find\_center \textit{(function)}}
    \vspace{0.5ex}

    \begin{boxedminipage}{\textwidth}

    \raggedright \textbf{find\_center}(\textit{region})

    \vspace{-1.5ex}

    \rule{\textwidth}{0.5\fboxrule}
\begin{alltt}
Compute the center of a star using MPC algorithm. 
Based on 'mpc\_cntr' from IRAF imutil task 'center'
in the cl.proto package.

Syntax:
    center = find\_center(region)
Input:
    region - slice of array around target star
Output: 
    center - array position of center as (y,x)
                relative to region origin\end{alltt}

    \vspace{1ex}

    \end{boxedminipage}

    \label{multireg:findobjects:get_positions}
    \index{multireg.findobjects \textit{(module)}!get\_positions \textit{(function)}}
    \vspace{0.5ex}

    \begin{boxedminipage}{\textwidth}

    \raggedright \textbf{get\_positions}(\textit{input}, \textit{sigma}=\texttt{2\-.\-0\-}, \textit{size}=\texttt{N\-o\-n\-e\-}, \textit{offset}=\texttt{N\-o\-n\-e\-}, \textit{region}=\texttt{N\-o\-n\-e\-}, \textit{thin}=\texttt{F\-a\-l\-s\-e\-})

    \vspace{-1.5ex}

    \rule{\textwidth}{0.5\fboxrule}
\begin{alltt}
Process the input array to return the list of positions that correspond
to all objects.

This function relies on Numarray nd\_image module for its operations.
Syntax:
    poslist,object,raw = get\_positions(input,offset=(0.,0.),region=None)
    
Input: 
    input   :  numarray array of the (wavelet scaled?) science data
                this array should correspond to slice specified in 'region'
    sigma   : sigma for gaussian for source/edge detection in image
    size    :  detection limit for objects, typically the size of the kernel
                used to filter input image. If None, no minimum size
                will be imposed. Default: None
    offset  : pixel offset of chip relative to final output frame
    region  : slice from full frame array to be used for object detection
    thin    : use homotropic thinning algorithm to extract line segments 
                corresponding to detected edges, rather than border of 
                identified edge region. This algorithm currently runs
                MUCH SLOWER than border extraction.
            
Output:
    poslist,objects,rawlist
    where
        poslist: list of positions for identified objects
        objects: list of slices corresponding to identified objects
        rawlist: list of positions for identified objects from array
        
    poslist and rawlist are of the form:
        [id, position, mag]
    where:
        id      :  target number from chip (integer)
        position:  list of position(s) given as x,y pairs (list of lists)
        counts  :  photometry for position(s)(float): 
                    sum of masked region for sources, mean value for edges
    The positions reported in these lists correspond either to a single 
    position for a positive source (star,small galaxy,...) or a list of 
    x,y pixels which correspond to edge features in image.\end{alltt}

    \vspace{1ex}

    \end{boxedminipage}

    \label{multireg:findobjects:LOG_function}
    \index{multireg.findobjects \textit{(module)}!LOG\_function \textit{(function)}}
    \vspace{0.5ex}

    \begin{boxedminipage}{\textwidth}

    \raggedright \textbf{LOG\_function}(\textit{x}, \textit{y}, \textit{s})

    \vspace{-1.5ex}

    \rule{\textwidth}{0.5\fboxrule}
    Return Laplacian-of-Gaussian for a given position with a width of 
    sigma s.

    \vspace{1ex}

    \end{boxedminipage}

    \label{multireg:findobjects:RADTODEG}
    \index{multireg.findobjects \textit{(module)}!RADTODEG \textit{(function)}}
    \vspace{0.5ex}

    \begin{boxedminipage}{\textwidth}

    \raggedright \textbf{RADTODEG}(\textit{rad})

    \end{boxedminipage}

    \index{multireg.findobjects \textit{(module)}|)}
